\subsection{Two methods: scale and detail}

VIPUR is a machine learning approach for predicting pathogenicity of proteins. The 106 features that were used for machine learning originate mainly (94\%) from the Rosetta software suite (Section \ref{subsec:MM_Rosetta}) applications; DDG monomer (Section \ref{subsubsec:MM_DDG_Monomer}), Relax (Section \ref{subsubsec:MM_Relax}) and Rescore (Section \ref{subsubsec:MM_Rescore}), the remaining features were collected from PSI-BLAST (Section \ref{subsec:MM_PSI_BLAST}) and Probe (Section \ref{subsec:MM_Probe}). 
All proteins in the VTS of which structures were known or had fragments available were collected from Modbase \cite{} and
% modbase, a database of annotated comparative protein structure models and associated resources, pieper
SWISS-MODEL \cite{}.
% Automated comparative protein structure modeling with SWISS-MODEL and Swiss-PdbViewer: A historical perspective, Guex
% Modeling protein quaternary structure of homo- and hetero-oligomers beyond binary interactions by homology, Bertoni
% Toward the estimation of the absolute quality of individual protein structure models, Benkert
% The SWISS-MODEL Repository—new features and functionality, Bienert
% SWISS-MODEL: homology modelling of protein structures and complexes, Waterhouse
Proteins that did not have a structure within the databases were modeled with Modeller (Section \ref{subsec:MM_Modeller} based on protein fragments that had the highest amino acid sequence identity to the protein.
In some experimental determined structures duplicate chains, ligands, metals and non-standard amino acids were present, these inconsistencies are able to alter the features generated by software and could in some case hinder feature collection, therefor they were removed to make the data homogeneous. Structural mutations of proteins that are in the VTS were introduced by a script using PyMOL (Section \ref{subsec:MM_PyMOL}) by default or PyRosetta (Section \ref{subsec:MM_PyRosetta}) if PyMOL was not available.

Another approach for determining pathogenicity of a mutation is by assessing energy differences between a wild type and mutant protein residues inside its complex. Analyzing mutations from this perspective gives the ability to view a complex in whole and determine how residues cause perturbations in a complex. Missense mutations in monomers of complexes were made with Modeller (Section \ref{subsec:MM_Modeller}) and the backbone was refined with Rosetta's backrub application (Section \ref{subsubsec:MM_Backrub}), to lower the energy levels within side chains Rosetta relax (Section \ref{subsubsec:MM_Relax}). This method shows similarities to that of VIPUR,  was tested with TNFRSF1A (Section \ref{subsec:CD_TNFRSF1A}) and its ligands TNF $\alpha$ and $\beta$. This method keeps: duplicate chains ligands and metals within the structure, water is excluded since it can cause issues with Rosetta tools (Section \ref{subsec:MM_Rosetta:}).

\subsection{Rosetta}
% DDG MONOMER CAREFULL STILL NEEDS TO BE LOOKED AT IF IT IS USEFUL AL.
Rosetta is a software suite that has a variety of tools that are developed to aid in macro molecular and antibody ,analysis, design and modeling \cite{}.
% About | RosettaCommons, Rosetta Commons
Both approaches rely on the Relax (Section \ref{subsubsec:MM_Relax}) for minimizing side chains and on DDG monomer (Section \ref{subsubsec:MM_DDG_Monomer}) to determine energy differences within the mutated protein. VIPUR uses rescore (Section \ref{subsubsec:MM_Rescore}) to acquire information about protein structures.

Both methods rely on Relax  to minimize energies in the side chains of the remodeled structures. With  DDG monomer  both rely on energy minimization's  in the side chains of the protein structures and need to information on energy changes in 

The scores generated for the machine learning within the VIPUR approach rely on results generated by Rosetta software and to apply this approach the steps are reproduced.  
Several strategies were employed for realizing mutated structures, the first strategy was to identify the whole structure of proteins


 The initial structure of the protein was produced with the application abinitio relax. For the prediction the application requires an amino acid sequence to identify homologous sequences in a curated database. Homologous sequences within the database are found by the BLAST algorithm, when a

For the search of the sequences it uses the BLAST algorithm and to find homologous amino acid sequences which have protein structures.

requires an amino acid sequence and  it takes an amino acid sequence as input and searches in a curated protein database BLAST for finding homologous sequences. 


to align sequences with to acquire homologous sequences.  The homologous With these sequences it finds structures related to the protein
For the prediction of the initial structure of TNFR the application abinitio relax was used. 

With this tool a sequence is inserted as input that is aligned to 


Missense mutated proteins have an altered amino acid that can cause differences in interactions with other amino acids, which can influence the backbone or side chain positions of a protein and therefore affect the structure. Software that makes missense mutations in protein structures (Modeller, PyMOL, PyRosetta) tend to replace residues without optimizing, causing odd energy levels or steric hindrance to arise.

\subsubsection{Relax}
\label{subsubsec:MM_Relax}

\subsubsection{DDG Monomer}
\label{subsubsec:MM_DDG_Monomer}

\subsubsection{Rescore}
With this tool Rosetta scores can be calculated based on silent or PDB files proteins structures \cite{} , the output is identical to that is written within the score files produced by Relax (Section \ref{subsubsec:MM_Relax}).
%Score Commands, Jared
\label{subsubsec:MM_Rescore}

\subsubsection{Abinitio}
\label{subsubsec:MM_Abinitio}

\subsubsection{Backrub}
The backrub application alters a protein by moving the backbone residues which it does with a strategy called fix end move (FEM). With this strategy, groups of residues are made at random that can contain up to: four dihedral, two bond angles and two end points.
Both ends are fixated at their position in which a new angle $\alpha$ is setup, within this angle residues are pivoted $\pm \ang{30}$ from the end points. When group are made and they only contain a single residue a different method is applied called pivot movement (PM). This method only rotates a single dihedral angle and could give a wider variety of results, most rotations lead to steric hindrance which makes it a process of trial and error.


 A partially avoided strategy is pivot movement (PM) which rotates a single dihedral of a residue, a downside of PM is that it 

 it could lead to better results but also has a high probability of causing steric hindrance in the structure. In some cases it is used if the group size used by FEM is 1.

In some cases a single residue is selected which would make FEM unsuitable, In some situations it is hard  To acquire information on energy in the groups it has to be known which residues are in the groups.


of which a new angle will be defined as $\alpha$ in which the residues will pivoted around to find the lowest energy. The 

 Each group's ends are fixa had to end points which are fixated at their position, the residues in between are pivoted to determine 

At the end of each group the residues are fixated and the residues in between

 Each group has two end points that are fixated and 

 of residues are made of which the end points in the structure are fixated at a single position, the residues that reside in between are pivoted.  

 fixates to residues at a single position and pivot 

can alter the backbone by moving the backbone residues, the strategy that is employed is fixed end moves (FEM) which w

 it with help of the Monte Carlo method (Section \ref{section:Chap_Monte_Carlo}). The application takes groups of residues from the structure and fixes the end points into a single position, all residues in between are pivoted until they reach an optimal point where the energy levels are low. Each pivot

 are taken and every residue in between is pivoted, compared to other methods that take a single residue element and pivot it which commonly would produce steric hindrance.

\label{subsubsec:MM_Backrub}

\label{subsec:MM_Rosetta}

\subsection{PyRosetta}
\label{subsec:MM_PyRosetta}

\subsection{BLAST}
\label{subsec:MM_BLAST}

\subsection{PSI-BLAST}
\label{subsec:MM_PSI_BLAST}

\subsection{Probe}
\label{subsec:MM_Probe}

\subsection{Modeller}
\label{subsec:MM_Modeller}

\subsection{GenomAD}
\label{subsec:MM_GenomAD}

\subsection{Infevers}
\label{subsec:MM_Infevers}

\subsection{RCSB}
\label{subsec:MM_RCSB}

\subsection{Uniprot}
\label{subsec:MM_Uniprot}

\subsection{PDB}
\label{subsec:MM_PDB}


 


\subsection{I-TASSER}
\label{subsec:MM_I_TASSER}

\subsection{PyMOL}
Visualization of 3D structures, making images of proteins and putting the known orientations of monomeres in position  were done in PyMOL \cite{}.
% PyMOL | pymol.org, Schrödinger
Since some protein structures consist of multiple identical monomers they are left out of the structure and supplied with information about how the monomers are position to form the whole oligomer structure (Sections \ref{subsec:MM_RCSB}, \ref{subsec:MM_Uniprot}).
\label{subsec:MM_PyMOL}

