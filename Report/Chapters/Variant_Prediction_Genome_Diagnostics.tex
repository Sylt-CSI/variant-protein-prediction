\subsection{Mutations and its effects in the central dogma of molecular biology}
Within the human genome mutations occur continuously by internal and external factors that: insert, remove, substitute or alter the reading frame in a nucleotide sequence.
Mutations are not without consequences and can be protective \cite{harper_protective_2015}, benign or harmful by altering the deoxyribonucleic acid (DNA) order.
From a sequence of DNA genes are transcribed into ribonucleic acid (RNA) which can work as machinery or translates into an amino acid sequence to form a protein. 
Mutations outside a gene could lead to lowered or heightened transcription of a protein, when a mutation resides inside a gene it could lead to proteins that are unstable during or after formation, perform less optimal or are not functional \cite{nih_sickle_nodate,nih_cystic_nodate,nih_traps_2016}.
%{1}
%Protective alleles and modifier variants in human health and disease, Harper
%{2}
%TRAPS, NIH
%Sickle cell disease, NIH
%Cystic fibroses, NIH

\subsection{A general concept of structural levels within proteins and the effect of mutations}
The formation of protein structures is classified in different levels, distinctions are made based on bindings and structures that arise with the interaction of bonds. 
The order in which amino acids appear in a sequence is called the primary structure, in this level amino acids are only bound to each other by peptide bonds. 
Within a primary structure amino acids can form new peptide bonds between the N-terminus and C-terminus of an amino acid, with these bonds 3D structures are made called $\alpha$-helices and $\beta$-sheets that together make up the secondary structure.
The tertiary structure gives further rise to the 3D shape of a polypeptide by making disulfide bridges, ion and hydrogen -bonds, hydrophobic and hydrophilic -interactions between amino acids
By combining multiple tertiary structures the quaternary structure of a protein can be formed out of the mentioned bonds, bridges and interactions \cite{wikipedia_protein_2019, bennion_protein_2002}.
% Protein structure, Wikipedia
%Protein Conformation and Diagnostic Tests: The Prion Protein, Bennion

Mutations within proteins can have different effects to protein structures, often single missense mutations often have minimal effect on the backbone of a protein \cite{feyfant_modeling_2007,chothia_relation_1986} but can result in destabilization of the structure when assembled or can disrupt the active site. Frameshift mutations on the other hand can cause large differences in the primary structure and have therefore a higher chance of an altered sequence that leads to deformation or stop codon introduction \cite{ogura_frameshift_2001}.
%Modeling mutations in protein structures, Feyfant et al.
%The relation between the divergence of sequence and structure in proteins., Chothia and Lesk 
%A frameshift mutation in NOD2 associated with susceptibility to Crohn's disease, Ogura et. al.
\label{subsec:GD_Structural_Levels_and_Mutation_Effects}


\subsection{Addition of structural data to diagnosis and treatment in healthcare}
Acquiring information about DNA sequences depends on sequencing, which became cheaper over the years \cite{nih_cost_nodate}, and found its use in diagnosing patients within the healthcare sector \cite{ van_der_velde_gavin:_2017}.
%The Cost of Sequencing a Human Genome, NIH
%GAVIN: Gene-Aware Variant INterpretation for medical sequencing, van der Velde 
From the collected data by genome sequencing experiments most of the analysis is handled in-silico due to the quantities of data that are produced \cite{ van_der_velde_gavin:_2017}. Proteins often find their use in diagnosing diseases experimentally \cite{hortin_introduction:_2010, bennion_protein_2002}, however in-silico it is often limited to information about conservation in the amino acid sequence \cite{ng_sift:_2003}.
% Introduction: Advances in Protein Analysis for the Clinical Laboratory, Hortin
%Protein Conformation and Diagnostic Tests: The Prion Protein, Bennion
%{3}
%
Yet, the 3D shape of proteins defines their function \cite{nanev_how_2008} and by assessing structures it can become possible to determine changes in function that are caused by a mutations that might not be discoverable through conservation and are therefore unclassifiable.
%How do crystal lattice contacts reveal protein crystallization mechanism?, Nanev
Another advantage of structural information is that it becomes possible to develop treatment with diagnostic information for diseases that are caused by mutations \cite{niu_protein-structure-guided_2016}.
% Protein-structure-guided discovery of functional mutations across 19 cancer types, Niu 
With experimental methods such as X-ray crystallography and nuclear magnetic resonance (NMR) more than 158000 structures \cite{wwpdb_wwpdb:_nodate} have been completely revealed, however it is only a tiny fraction of the potential possible proteins \cite{ cantrill_chemiotics:_nodate} (especially without the inclusion of all folds).
% wwPDB: Deposition Statistics, wwPDB	
%Chemiotics: How many proteins can we make? : The Sceptical Chymist, Cantrill
Making 3D structures is currently not common for diagnosis because it is relative expensive and is difficult to perform, some structures contain flexible regions which makes it hard to determine the exact position of some atoms and can cause information loss about the structure \cite{pdb101_pdb101:_nodate, ridgen_protein_nodate}.
%PDB101: Learn: Guide to Understanding PDB Data: Missing Coordinates and Biological Assemblies, PDB101
%From Protein Structure to Function With Bioinformatics page 4 (book) 18 (pdf), Ridgen
\label{subsec:GD_Addition_of_structural_data}
\newpage
%Within DNA sequences of humans continuously mutations occur from internal and external factors, some of these mutation insert, remove or substitute nucleotides which alter the the rate of tr
% Some types of mutations can be repaired \cite{}, other mutations are made on purpose\cite{}, incorrectly repaired \cite{} or are irreparable by the available mechanisms \cite{} and will be carried over to the next generation of cells or even to the offspring.
%DNA repair, Wikipedia
%Janeway’s Immunobiology page 410-411 10.7: Germinal center B cells undergo V-region somatic hypermutation, and cells with mutations that improve af nity for antigen are selected, Kenneth murphy
% Repair of naturally occurring mismatches can induce mutations in flanking DNA, Chen et al.
% Error-Prone Repair of DNA Double-Strand Breaks, Rodgers and McVey


%
%Evidence is based on conservation whether it works or not
%
%Telling about the current state of gene variant prediction in the perspective of machine learning, GAVIN, SIFT, CADD

