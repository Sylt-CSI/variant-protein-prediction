All used methods use structural information acquired from previous experiments and make it therefore difficult to make predictions based on structural information because many of the structures are fragments.

At its current state VIPUR is not usable for diagnosing defects within proteins and severe changes should be made to make it usable. 

SPVAA is only helpful when an expert is available to determine if a mutation has any affect on the protein structure,  at its current state it is not user friendly or helpful to inexperience users, which is reflects in its results. CYS 62 GLY was the only mutation that without much further knowledge could be assumed pathogenic based on the plots and models (Figures \ref{fig:relax_TNFA_scores}, \ref{fig:relax_TNFB_scores}, \ref{fig:RES_TNFA_18}, \ref{fig:RES_TNFB_18}) that were produced, none of the other mutations contained clear information whether they would be pathogenic or not and hardly any differences have been observed between TNF$\alpha$ -$\beta$. 

HOPE is an informative tool that gave new insight in the mutation GLU 138 ALA, in some cases it can be very clear and almost form a conclusion but in other situations it is unable to discover effects of a mutation to elucidates its user and makes the dependence of previously investigated knowledge visible.
