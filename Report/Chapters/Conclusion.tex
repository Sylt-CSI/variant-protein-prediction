All used methods use structural information acquired from experimentally determine structures which are mostly fragments due to the difficulty and expenses of determining structures. Because the are mostly fragments and miss context it can be hard to make relevant assumptions about some protein structures or its general effect in a cell.

At its current state VIPUR is not usable for diagnosing defects within proteins and most likely also not for predicting deleteriousness. To make it usable for predicting effects of variants or deleteriousness it could benefit of manual curation of all wild types, which is partially fulfilled by using proteins from the SWISS-MODEL database. A different addition would be looking at the structure in context of its environment since that is where the wild type should behave properly. With that information proper predictions can be made about the deleteriousness of a protein or its pathogenicity.  

SPVAA analyses proteins more into a natural context than VIPUR because it allows the use of more chains and ligands. However the assessment of structures made by SPVAA require expertise to determine the effects of mutations and at its current state it is not user friendly or helpful to inexperienced users. CYS 62 GLY was the only mutation that without prior knowledge could be assumed pathogenic based on the plots and models (Figures \ref{fig:relax_TNFA_scores}, \ref{fig:relax_TNFB_scores}, \ref{fig:RES_TNFA_18}, \ref{fig:RES_TNFB_18}) that were produced. None of the other mutations contained clear information whether they would be pathogenic or not and hardly any differences have been observed between TNF$\alpha$ -$\beta$. 

HOPE is an informative tool that gave new insight in the mutation GLU 138 ALA, in some cases it can be very clear and almost form a conclusion but in other situations it is unable to discover effects of a mutation to elucidates its user and makes the dependence of previously investigated knowledge visible.
