Ongeveer 1 op de 17 mensen is slachtoffer van een ~7.000 bekende zeldzame ziekte. De meeste van deze patiënten krijgen geen diagnose, wat betekent dat ze geen duidelijke prognose krijgen en niet in staat zijn om zich aan te sluiten bij specifieke patiëntenondersteuningsgroepen en daarmee geen geschikte behandeling krijgen.
Next-generation sequencing (NGS) van DNA geeft de mogelijheid tot moleculaire diagnose om deze patiënten te helpen, maar vele uitdagingen staan ​​nog steeds in de weg.
De afgelopen jaren zijn er grote vorderingen gemaakt in machine learning methoden die de hoeveelheid DNA-varianten verminderen die door menselijke experts worden vastgesteld als potentieel pathogeen.
Maar de huidige software die afhankelijk is van eigenschappen zoals evolutionair behoud, annotatie van regulatorische genomische-elementen en structurele DNA-kenmerken zijn over vele jaren geoptimaliseerd en er worden geen significante verbeteringen meer verwacht.
Hier hebben we geprobeerd structurele kenmerken van eiwitten in de diagnostiek te introduceren op basis van de methoden die door VIPUR werden gebruikt. Door de moeilijkheden van eiwitmodellering en de kennis van deskundigen hebben we ontdekt dat de door VIPUR gebruikte methoden niet bruikbaar zijn bij de diagnose met machine learning. Structurele gegevens van eiwitten zijn vaak onvolledig en zijn in hoge mate afhankelijk van experimenteel bepaalde structuren die duur zijn om te maken. De methoden die VIPUR gebruikt voor het standaardiseren van eiwitstructuren voor machine learning, verwijderen de context en behandelen eiwitten alsof ze in een vacuüm zitten. Om een ​​meer realistisch beeld te krijgen van structurele kenmerken hebben we ervoor gekozen om de webservice HOPE te gebruiken. Ook is er een methode ontwikkeld om inzicht te krijgen in de structurele kenmerken van een eiwit en zijn varianten. Met deze method hebben we geprobeerd om structurele informatie bruikbaar te maken voor diagnose.
