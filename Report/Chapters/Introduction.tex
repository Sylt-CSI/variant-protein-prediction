Around 1 in 17 people is affected by one of ~7,000 known rare diseases. Most of these patients do not receive a diagnosis, which means they remain in uncertainty without a prognosis, are unable join specific patient support groups, and do not receive the most appropriate treatment.
Next-generation sequencing (NGS) of DNA promises to establish a molecular diagnosis and help these patients but many challenges still stand in the way of maximum success.
Recent years have seen great advances in computational tools that quickly reduce the amount of DNA variants to be interpreted by a human expert for potentially pathogenic effects \cite{van_der_velde_gavin:_2017}. Although algorithms can now safely remove around 95\% of the harmless variants, this still leaves hundreds of variants to be investigated for a whole-exome sequenced patient, which is far too many for a quick and clear diagnosis. Current tools to predict variant pathogenicity rely on features such as evolutionary conservation, annotation of regulatory genomics elements or structural DNA features. These tools have already been optimized over many years and further significant improvements are not expected. Therefore there is still a great need for even more powerful variant prioritization tools. 
% 1,2
A refreshing alternative was presented by variant interpretation using Rosetta (VIPUR) \cite{baugh_robust_2016} which shows the potential of structural modeling of proteins to predict the actual effect of a specific variant on the function of a protein. However the publication did not result in a piece of high quality software that is usable for routine diagnostics. Therefore we want to test if the structural information used by VIPUR is useful to diagnostics with the difficult to assess gene tumor necrosis factor receptor 1 alpha (TNFRSF1a) to see if the prediction has any meaning. To test this we use two validated mutations of TNFRSF1A and one provisional mutation in combination with the tool have your protein explained (HOPE) \cite{venselaar_protein_2010} and a self developed method Simple Protein Variant Analyses Approach (SPVAA). Through testing these methods we will explore the potential pitfalls of protein modeling and discover if it is possible to add structural information to routine diagnostics.
