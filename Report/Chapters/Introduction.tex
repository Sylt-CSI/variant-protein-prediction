Around 1 in 17 people is affected by one of ~7,000 known rare diseases. Most of these patients do not receive a diagnosis, which means they remain in uncertainty without a prognosis, are unable join specific patient support groups, and do not receive the most appropriate treatment.
Next-generation sequencing (NGS) of DNA promises to establish a molecular diagnosis and help these patients but many challenges still stand in the way of maximum success.
    Recent years have seen great advances in computational tools that quickly reduce the amount of DNA variants to be interpreted by a human expert for potentially pathogenic effects. Although algorithms can now safely remove around 95\% of the harmless variants, this still leaves hundreds of variants to be investigated for a whole-exome sequenced patient, which is far too much for a quick and clear diagnosis.
Current tools to predict variant pathogenicity rely on indirect evidence such as evolutionary conservation, annotation of regulatory genomics elements or structural DNA features. A refreshing alternative was presented by VIPUR which shows the potential of structural modelling of proteins to predict the actual effect of a specific variant on the function of that protein. However, this predictor was not integrated with the latest and greatest variant pathogenicity prediction approaches, was done on relatively small number of variants, and did not result in a tool that is ready to be taken into routine diagnostic practice.