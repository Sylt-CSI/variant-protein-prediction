\subsection{Levels of Protein Structures}
The formation of protein structures is classified in different layers, distinctions are made based on bindings and structures that arise from them. 
The order in which amino acids appear in a sequence only bound to each other by peptide bonds is called the primary structure. 
Within an amino acid sequence new peptide bonds can be formed between the N-terminus and C-terminus which creates structures called $\alpha$ helices and $\beta$-sheets that together make up the secondary structure.
Some amino acids that reside within a sequence are able to form: disulfide bridges, ion, hydrogen -bonds, hydrophobic and hydrophilic -interactions that are able to attract or repulse other amino acids within the same sequence and form with that a new structure called the tertiary structure.
By combining multiple tertiary structures together new bonds can be formed from the previously mentioned types of bonds and interactions, together the structures give rise to the quaternary structure.

what makes it so complex 

Importance of experimental determination instead of just modeling

Modeling protein techniques as homology modeling and protein threading

Simulations such as gromacs 