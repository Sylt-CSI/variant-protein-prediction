\subsection{A general concept of structural levels within proteins}

The formation of protein structures is classified in different levels, distinctions are made based on bindings and structures that arise from them. 
The order in which amino acids appear in a sequence is called the primary structure, in this level amino acids are only bound to each other by peptide bonds. 
Within a primary structure amino acids can form new peptide bonds between the N and C -terminus of an amino acid, with these bonds 3D structures are made called $\alpha$ helices and $\beta$-sheets that together make up the secondary structure.
More alterations to a single amino acid sequence in the 3D can come from disulfide bridges, ion, hydrogen -bonds, hydrophobic and hydrophilic -interactions formed by the residues of the amino acids, together these bonds form the tertiary structure.
By combining multiple tertiary structures the quaternary structure of a protein can be formed out of the mentioned bonds, bridges and interactions.

\subsection{Problems in protein modeling and strategies for resolving structures}
At the moment of writing more than 158000 \cite{} protein structures have been resolved by experimental methods such a X-ray crystallography and nuclear magnetic resonance (NMR). 
% wwPDB: Deposition Statistics, wwPDB	
Importance of experimental determination instead of just modeling

Modeling protein techniques as homology modeling and protein threading

Simulations such as gromacs 