\subsection{Apoptosis within humans}
Each human has about 37.2 trillion cells (3.72 x 10\textsuperscript{13}) \cite{} of which several types are relative short lived \cite{} compared to the life expectancy of a human in 2016 \cite{}. 
%An estimation of the number of cells in the human body, Bianconi et.al
% Cell Biology by the Numbers, chapter 4 Rates and Duration, page 279, Table 4-9, Ron and Rob
%Life expectancy at birth, total (years) | Worldbank
Continuously cells die by programmed cell death which is called apoptosis, this process allows to make certain features arise and keep cell growth in check \cite{}.
%Molecular Biology of THE CELL chapter 18 Cell Death, page 1021-1023 (book) 1056-1058 (pdf)
The process of apoptosis can be triggered by pathways that activate caspases (proteases that cleave aspartate in proteins), once the process starts it is irreversible and the amount of caspases within the cell increases and is going disrupt the cells metabolism \cite{}.
%Molecular Biology of THE CELL, chapter 18 Cell Death, the Intrinsic pathway of Apoptosis Depends on Mitochondria, page 1025-1028 (book) 1060-1063 (pdf)
The internal system that determines when apoptosis initiates is the intrinsic pathway, it activates when their is internal stress in the cell such as damaged DNA or proteins (Which can caused by: heat, hypoxia, radiation, low/high ion concentration within a cell.)  \cite{}. 
%Robbins and Cotran Pathologic Basis of Disease, Professional Edition, chapter 7, figure 7-33, page 302 (book) 321 (pdf), Vinay et.al
%Molecular Biology of THE CELL, chapter 18 Cell Death, the Intrinsic pathway of Apoptosis Depends on Mitochondria, page 1025-1028 (book) 1060-1063 (pdf)
If stress is detected a mitochondrion releases cytochrome c into the cytosol and triggers a cascade, cytochrome c binds to apoptic protease activating factor 1 (APAF1) and starts to activate (intiator) caspase 9 that activates (executioner) caspase 3 and thereby destroying proteins structures within the cell \cite{}. 
%Molecular Biology of THE CELL, page 1021-1028, Bruce et al.

Not only internal signals within a cell are able to trigger apoptosis, neighboring cells are able to produce cytokines that activate the extrinsic pathway of apoptosis. One cytokine that the mainly initiates apoptosis within cells is the Fas ligand (FasL), once a cytotoxic T lymphocyte attaches its FasL on a Fas receptor on the cell surface it will start to recruit Fas associated death domain proteins (FADD) in the cytoplasm and assemble it into the death-inducing signaling complex (DISC) that activates (initiator) caspase 8 and activates (executioner) caspase 3 \cite{}.
%Robbins and Cotran Pathologic Basis of Disease, Professional Edition, chapter 2 Morphologic and Biochemical Changes in Apoptosis, Page 56 (Book) 75 (pdf), Vinay
%Molecular Biology of THE CELL chapter 18 Cell Death, Cell-Surface Death receptors Activate the Extrinsic pathway of Apoptosis, page 1024-1025 (book) 1059-1060 (pdf)
Two different cytokines that can activate a less direct form of the extrinsic pathway are tumor necrosis factor $\alpha$ and $\beta$ (TNF) (Section \ref{subsec:CD_TNF_A_B}) by interacting with the tumor necrosis factor receptor 1 $\alpha$ (TNFRSF1A, TNFR1) (Section \ref{subsec:CD_TNFRSF1A}). On binding of TNF to TNFR1 the silencer of death domain (SODD) proteins release on the cytoplasmic site and TRADD will start to bind together with proteins that will form a complex that attracts FADD and will continue to follow the identical pathway as with the FasL \cite{}.
%Robbins and Cotran Pathologic Basis of Disease, Professional Edition, chapter 2 Necroptosis,  Page 59 (Book) 78 (pdf), Vinay
%TNF-R1 Signaling: A Beautiful Pathway, Chen and Goeddel
\label{subsec:CD_cell_death}

\subsection{Tumor Necrosis Factor Receptor Associated Syndrome}
Tumor necrosis factor receptor-associated periodic syndrome (TRAPS) is classified as a rare disease (1 : 1,000,000) and was formerly known as Familial Hibernian fever (FHF) \cite{}, is a hereditary autosomal dominant disease which can cause recurring fevers with a duration from days up to several months. Symptoms during these fevers are: skin rash, swelling, inflammatory reactions across the whole body and pain in the abdomen, muscles and/or joints, a long term and lasting effect is the accumulation of amyloid within the kidneys and may result in other diseases \cite{}. 
%Tumor Necrosis Factor Receptor Associated Periodic Syndrome (Juvenile), Roth-Wojcicki
%TRAPS, NIH
TRAPS is known to be caused by mutations within the gene TNFRSF1A (Section \ref{subsec:CD_TNFRSF1A}),which is part of the extrinsic pathway (Section \ref{subsec:CD_cell_death}), the mutated proteins tend to get trapped in the cell and will be unable to reach the cell surface and therefore start activating a inflammatory response \cite{}.
% Falling into TRAPS--receptor misfolding in the TNF receptor 1-associated periodic fever syndrome, Kimberly
%TRAPS, NIH
So far 158 mutations have been associated with the disease \cite{}, but more mutations have been identified in TNFRSF1A wherein some might be pathogenic (Sections \ref{subsec:MM_GAVIN_data_table}, \ref{subsec:MM_GenomAD}).
%Tabular list ,Aksentijevich
\label{subsec:CD_TRAPS}

\subsection{Tumor Necrosis Factor Receptor Super Family Member 1A}
Tumor Necrosis Factor Receptor Super Family Member 1A (TNFRSF1A, TNFR1) is a gene located on chromosome 12 region 1 band 3 and sub-band 31. The gene produces a trans-membrane receptor consisting of 445 residues divided into 221 residue cytoplasmic section and a 171 extracellular part that consists of 4 conserved cysteine rich domains \cite{}.
% Cloning of human tumor necrosis factor (TNF) receptor cDNA and expression of recombinant soluble TNF-binding protein, Gray
%Encyclopedia of Molecular Pharmacology page 1247-1251 book, 1240-1244 pdf, Walter Rosenthal
%Crystal structure of the soluble human 55 kd TNF receptor-human TNFβ complex: Implications for TNF receptor activation, Banner
The receptor is ubiquitous across most cell surfaces ,but not on erythrocytes \cite{}, and can form two different types of unbound hexagonal clusters depending on the dimer formation \cite{}. 
%Innate myeloid cell TNFR1 mediates first line defence against primary Mycobacterium tuberculosis infection, Segueni 
%Crystallographic Evidence for Dimerization of Unliganded Tumor Necrosis Factor Receptor, Naismith
When the structures are dimers the binding sites are exposed and make it possible for TNF (Section \ref{subsec:CD_TNF_A_B}) to bind in trimeric form, with binding of TNF the dimers disconnect and three TNFR1s connect to TNF trimer \cite{}.
%Crystallographic Evidence for Dimerization of Unliganded Tumor Necrosis Factor Receptor, Naismith
When TNF is bound it will activate TRADD \cite{} and initiate the caspase cascade \cite{} (Section \ref{subsec:CD_cell_death}).
%The adaptor protein TRADD activates distinct mechanisms of apoptosis from the nucleus and the cytoplasm, Bender
%The biochemistry of apoptosis, Hengartner
\label{subsec:CD_TNFRSF1A}

\subsection{Tumor Necrosis Factor Alpha and Beta}
The proteins TNF $\alpha$ and $\beta$ are both pro-inflammatory cytokines that are produced as response to an infection or when a cell is damaged. Both are transcribed from their genes that reside in chromosome 6 in the p-arm at region 2 band 1 and sub-band 3. TNF $\alpha$ and $\beta$ are 35\% identical and 50\% homologous to each other consisting out of 233 and 205 amino acid residues. Both are able to form a homotrimeric structures that can bind to the dimeric form TNFR1 (Section \ref{subsec:CD_TNFRSF1A}) to activate the extrinsic pathway\cite{}. 
%Characterization of receptors for human tumour necrosis factor and their regulation by γ-interferon, Aggarwal
%OMIM Entry - * 153440 - LYMPHOTOXIN-ALPHA; LTA, Hamosh
%A novel form of TNF/cachectin is a cell surface cytotoxic transmembrane protein: Ramifications for the complex physiology of TNF, Kriegler
%Encyclopedia of Molecular Pharmacology page 1247-1251 book, 1240-1244 pdf, Walter Rosenthal
\label{subsec:CD_TNF_A_B}