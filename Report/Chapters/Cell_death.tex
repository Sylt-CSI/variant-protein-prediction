\subsection{Tumor Necrosis Factor Receptor Associated Syndrome}
Tumor necrosis factor receptor-associated periodic syndrome (TRAPS) is classified as a rare disease (1 : 1,000,000) and was formerly known as Familial Hibernian fever (FHF) \cite{}, is a hereditary autosomal dominant disease which can cause recurring fevers with a duration from days up to several months. Symptoms during these fevers are: skin rash, swelling, inflammatory reactions across the whole body and pain in the abdomen, muscles and/or joints, a long term and lasting effect is the accumulation of amyloid within the kidneys and may result in other diseases \cite{}. 
%Tumor Necrosis Factor Receptor Associated Periodic Syndrome (Juvenile), Roth-Wojcicki
%TRAPS, NIH
TRAPS is known to be caused by mutations within the gene turmor necrosis factor receptor 1 (TNFRSFA1/TNRF1) (Section \ref{subsec:CD_TNFRSF1A}), the mutated proteins tend to get trapped in the cell and will be unable to reach the cell surface and therefore start activating a inflammatory response \cite{}.
% Falling into TRAPS--receptor misfolding in the TNF receptor 1-associated periodic fever syndrome, Kimberly
%TRAPS, NIH
So far 158 mutations have been associated with the disease \cite{}, but more mutations have been identified in TNFRSF1A wherein some might be pathogenic (Sections \ref{subsec:MM_GAVIN_data_table}, \ref{subsec:MM_GnomAD}).
%Tabular list ,Aksentijevich
\label{subsec:CD_TRAPS}

\subsection{Tumor Necrosis Factor Receptor Super Family Member 1A}
Tumor Necrosis Factor Receptor Super Family Member 1A (TNFRSF1A, TNFR1) is a gene located on chromosome 12 region 1 band 3 and sub-band 31. The gene produces a trans-membrane receptor consisting of 445 residues divided into 221 residue cytoplasmic section and a 171 extracellular part that consists of 4 conserved cysteine rich domains \cite{}.
% Cloning of human tumor necrosis factor (TNF) receptor cDNA and expression of recombinant soluble TNF-binding protein, Gray
%Encyclopedia of Molecular Pharmacology page 1247-1251 book, 1240-1244 pdf, Walter Rosenthal
%Crystal structure of the soluble human 55 kd TNF receptor-human TNFβ complex: Implications for TNF receptor activation, Banner
The receptor is ubiquitous across most cell surfaces ,but not on erythrocytes \cite{}, and can form two different types of unbound hexagonal clusters depending on the dimer formation \cite{}. 
%Innate myeloid cell TNFR1 mediates first line defence against primary Mycobacterium tuberculosis infection, Segueni 
%Crystallographic Evidence for Dimerization of Unliganded Tumor Necrosis Factor Receptor, Naismith
When the structures are dimers the binding sites are exposed and make it possible for tumor necrosis factor (TNF) $\alpha$ and $\beta$  (Section \ref{subsec:CD_TNF_A_B}) to bind in trimeric form, with binding of TNF the dimers disconnect and three TNFR1s interact with the TNF trimer \cite{}.
%Crystallographic Evidence for Dimerization of Unliganded Tumor Necrosis Factor Receptor, Naismith
With the interaction of the TNF trimers with TNFR1 it can activate several pathways such as; the nuclear factor kappa-light-chain-enhancer of activated B cells (NF-$\kappa$B), which enhances the transcription of various genes during inflammation, infection or other forms of external stress; also it is able to activate the extrinsic pathway of apoptosis after binding of TNF to TNFR1, by releasing the silencer of death domain (SODD) proteins release on the cytoplasmic site. Tumor Necrosis Factor Receptor type 1-Associated DEATH Domain protein (TRADD) \cite{}
%The adaptor protein TRADD activates distinct mechanisms of apoptosis from the nucleus and the cytoplasm, Bender
 will start to bind together with proteins that will form a complex which will attract Fas associated death domain (FADD) and after two hours\cite{} if not inhibited. 
%Life And Death Decisions: Secondary Complexes and Lipid Rafts in TNF Receptor Family Signal Transduction,Muppidi
On binding of FADD initiator caspase 8  starts a cascade wherein caspase 3 is activated an will cleave aspartate out of proteins and thereby disrupting the metabolism\cite{}.
%Robbins and Cotran Pathologic Basis of Disease, Professional Edition, chapter 2 Necroptosis,  Page 59 (Book) 78 (pdf), Vinay
%TNF-R1 Signaling: A Beautiful Pathway, Chen and Goeddel
%The biochemistry of apoptosis, Hengartner
\label{subsec:CD_TNFRSF1A}

\subsection{Tumor Necrosis Factor Alpha and Beta}
The proteins TNF $\alpha$ and $\beta$ are both pro-inflammatory cytokines that are produced as response to an infection or when a cell is damaged. Both are transcribed from their genes that reside in chromosome 6 in the p-arm at region 2 band 1 and sub-band 3. TNF $\alpha$ and $\beta$ are 35\% identical and 50\% homologous to each other consisting out of 233 and 205 amino acid residues. Both are able to form a homotrimeric structures that can bind to the dimeric form TNFR1 (Section \ref{subsec:CD_TNFRSF1A}) to activate the extrinsic pathway\cite{}. 
%Characterization of receptors for human tumour necrosis factor and their regulation by γ-interferon, Aggarwal
%OMIM Entry - * 153440 - LYMPHOTOXIN-ALPHA; LTA, Hamosh
%A novel form of TNF/cachectin is a cell surface cytotoxic transmembrane protein: Ramifications for the complex physiology of TNF, Kriegler
%Encyclopedia of Molecular Pharmacology page 1247-1251 book, 1240-1244 pdf, Walter Rosenthal
\label{subsec:CD_TNF_A_B}