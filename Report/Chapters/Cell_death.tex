\subsection{Cell Death}
Each human has about 37.2 trillion cells (3.72 x 10\textsuperscript{13}) \cite{} of which several types are relative short lived \cite{} compared to the life expectancy of a human in 2016 \cite{}. 
%An estimation of the number of cells in the human body, Bianconi et.al
% Cell Biology by the Numbers, chapter 4 Rates and Duration, page 279, Table 4-9, Ron and Rob
%Life expectancy at birth, total (years) | Worldbank
Continuously cells die by programmed cell death which is called apoptosis, this process allows to make certain features arise and keep cell growth in check \cite{}.
%Molecular Biology of THE CELL, page 1021-1028, Bruce et al.
The process of apoptosis can be triggered by pathways that activate caspases (proteases that cleave aspartate in proteins), once the process starts it is irreversible and the amount of caspases within the cell increases and is going disrupt the cells metabolism \cite{}.
%Molecular Biology of THE CELL, page 1021-1028, Bruce et al.
The internal system that determines when apoptosis initiates is the intrinsic pathway, it activates when their is internal stress in the cell such as damaged DNA or proteins (Which can caused by: heat, hypoxia, radiation, low/high ion concentration within a cell.)  \cite{}. 
%Robbins and Cotran Pathologic Basis of Disease, Professional Edition, chapter 7, figure 7-33, page 302 (book) 321 (pdf), Vinay et.al
If stress is detected a mitochondrion releases cytochrome c into the cytosol and triggers a cascade, cytochrome c binds to apoptic protease activating factor 1 (APAF1) and starts to activate (intiator) caspase 9 that activates caspase 3 and thereby destroying proteins structures within the cell \cite{}. 
%Molecular Biology of THE CELL, page 1021-1028, Bruce et al.



%
%Other cells within a multi cellular organism are able to trigger the initiation of apoptosis with the extrinsic pathway. Once a cell excretes signal that it is not healthy 
%Another pathway that can be initiate is cause by signal from different cells 
%
%
%which triggers the release of cythocrome c from the mitochondria that binds to apoptic protease activating factor 1 (APAF1) and activates the initiator caspase 9 .  \cite{}.
%%Molecular Biology of THE CELL, page 1021-1028, Bruce et al.
%
%
%A pathway that can controls apoptosis internally is called the intrinsic pathway and is triggered by internal stress such as damage to DNA and proteins which can be caused by: heat, hypoxia, radiation, low/high ion concentration within a cell, \cite{}
%
%
%
%Within the process of apoptosis initiator caspases (proteases that cleave aspartate from proteins) are activated by factors that can trigger pathways that initiate apoptosis \cite{}.
%
%
%
%
%Most cell types are capable of apoptosis when triggered by internal or external factors and  
% known as apoptosis, to allow certain features to arise and keep cell growth in check.  Within the process of apoptosis caspases , a family of proteases that focuses on cleaving aspartate in proteins \cite{??????????},  play a major role in breaking organelles and structures in a cell. Apoptosis 
%
%Apoptosis can be triggered by internal and external factors; or infection. \cite{?????}
%%Molecular Biology of THE CELL, page 1021-1028, Bruce et al.
%
%
%Apoptosis can be triggered by internal and external factors; 
%
% is mainly triggered by two separate pathways that, the intrinsic pathway; which is a response internally from cell stress such as DNA damage or 
%
%When apoptosis is triggered it irreversible and amplifies its effect.
%
%Death of cells is a continuous process that happens within every human. One form that is necessary to allow certain functions and to keep cell growth in check is apoptosis, this form of regulated cell death is triggered by 
%
%
%
%to behave properly within an individual is apoptosis which the cells within a multi cellular organism intact this form is regulated and called apoptosis. With 
%
%
%  to occurs mainly in two forms; the unregulated form, which is defined as necrosis \cite{??????}, which is generally caused by external factors that damage a cell and results in rupturing cells that spill organelles and cytoplasm. The regulated form is ,known as apoptosis \cite{????????} , which can be triggered by two known separate pathways both resulting in cell death. The intrinsic pathway is triggered by 
%
% or by receiving signals from other cells (extrinsic pathway) \cite{}. 
%%Cell death: critical control points,Danial
%
%billions die everyday by the mec apoptosis. Apoptosis is a regulated form of cell death which is induced by  an unregulated manner is described as necrosis, parts of the cells can be missing or destory due to necrosis and apoptosis.  already 50 to 70 billion (50-70 x \textsuperscript{9}) die every day \cite{} by the process of apoptosis. 
%
%%Chapter 2.3 - Apoptosis, Growth, and Aging, Cole
%T
%Cells within a multi cellular organism go through several stages depending on their function, but the last stage is always cell death. 
%
%
%This can have multiple causes, but a regulated and common form of programmed cell death is apoptosis and is highly important fo


 are possible for activating the process and are caused by separate pathways. 
Both pathways lead to the activation of death-inducing signaling complex (DISC). This process is dependent on several 
\label{subsec:CD_cell_death}

\subsection{Tumor Necrosis Factor Receptor Associated Syndrome}
Tumor necrosis factor receptor-associated periodic syndrome (TRAPS) is classified as a rare disease (1 : 1,000,000) and was formerly known as Familial Hibernian fever (FHF) \cite{}, is a hereditary autosomal dominant disease which can cause recurring fevers with a duration from days up to several months. Symptoms during these fevers are: skin rash, swelling, inflammatory reactions across the whole body and pain in the abdomen, muscles and/or joints, a long term and lasting effect is the accumulation of amyloid within the kidneys and may result in other diseases \cite{}. 
%Tumor Necrosis Factor Receptor Associated Periodic Syndrome (Juvenile), Roth-Wojcicki
%TRAPS, NIH
TRAPS is known to be caused by mutations within the gene TNFRSF1A (Section \ref{subsec:CD_TNFRSF1A}),which is part of the extrinsic pathway (Section \ref{subsec:CD_cell_death}), the mutated proteins tend to get trapped in the cell and will be unable to reach the cell surface and therefore start activating a inflammatory response \cite{}.
% Falling into TRAPS--receptor misfolding in the TNF receptor 1-associated periodic fever syndrome, Kimberly
%TRAPS, NIH
So far 158 mutations have been associated with the disease \cite{}, but more mutations have been identified in TNFRSF1A wherein some might be pathogenic (Sections \ref{subsec:MM_GAVIN_data_table}, \ref{subsec:MM_GenomAD}) .
%Tabular list ,Aksentijevich
\label{subsec:CD_TRAPS}

\subsection{Tumor Necrosis Factor Receptor Super Family Member 1A}
Tumor Necrosis Factor Receptor Super Family Member 1A (TNFRSF1A) is a cell membrane receptor consisting of 445 residues divided into a 171 extracellular and a 221 residue cytoplasmic section \cite{}.
% Cloning of human tumor necrosis factor (TNF) receptor cDNA and expression of recombinant soluble TNF-binding protein, Gray
On the extracellular side of the protein tumor necrosis factor (TNF) $\alpha$ and $\beta$ are able to bind in trimeric form, by binding it activates TRADD \cite{} and initiates caspase 8 \cite{}.
%The adaptor protein TRADD activates distinct mechanisms of apoptosis from the nucleus and the cytoplasm, Bender
%The biochemistry of apoptosis, Hengartner

dimers? 
%Crystallographic Evidence for Dimerization of Unliganded Tumor Necrosis Factor Receptor
\label{subsec:CD_TNFRSF1A}

\subsection{Tumor Necrosis Factor Alpha and Beta}
\label{subsec:CD_TNF_A_B}