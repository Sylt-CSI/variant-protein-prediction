The VIPUR approach could be investigated to test whether the features generated by PSI-BLAST are the main predictors. To measure PSI-BLAST feature importance within VIPUR it first needs to be reverse engineered to make it work with Modeller or another tool which is able to implement mutations in PDB files. Once the reverse engineering is finished VIPURs feature importance can be tested with shap \cite{slundberg_slundberg/shap:_nodate, strumbelj_explaining_2014, ribeiro_why_2016, shrikumar_learning_2017, datta_algorithmic_nodate, bach_pixel-wise_2015, datadive_interpreting_nodate} that explains the output of machine learning models and or with similar methods and software on the VTS.
%slundberg/shap: A unified approach to explain the output of any machine learning model, Slundberg
%Explaining Prediction Models and Individual Predictions with Feature Contributions, Štrumbelj
%Why Should I Trust You?": Explaining the Predictions of Any Classifier, Ribeiro
%Learning Important Features Through Propagating Activation Differences, Shrikumar
%Algorithmic Transparency via Quantitative Input Influence: Theory and Experiments with Learning Systems, Datta
%On Pixel-Wise Explanations for Non-Linear Classifier Decisions by Layer-Wise Relevance Propagation, Bach
%Interpreting random forests | Diving into data, Datadive

SPVAA is currently highly dependent on the resources that are available and comes short to produce enough models \cite{rosetta_commons_analyzing_nodate}. One option to improve the quantity of models is by using different software that is less resource intensive. Another option would be to run SPVAA on a different cluster that has more nodes and allows to setup jobs that use multiple nodes.  SPVAA is not a prediction method and has to be modified and expanded become a variant predictor that is able to predict pathogencity or deleteriouness. A good starting point for such a predictor would be according to the guidelines in section \ref{subsec:GD_theoratical_large_scale_implementation}.

% MOLECULAR DYNAMICS
VIPUR and SPVAA could both be improved in various ways, one of them would be by doing molecular dynamic simulations on the mutated structures to determine the effects of structural changes. With SPVAA it would have most likely become clear if the loss of the disulfide bridge from CYS 62 GLY  in TNFRSF1A would have caused structural issues. VIPUR could benefit from molecular dynamics as new machine learning feature in situations where limited movement is observed and it changes tremendously when a missense mutation occurred in a protein increases with a mutation or vice versa.



