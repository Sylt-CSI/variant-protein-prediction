The VIPUR approach will be further investigated to test whether the features generated by PSI-BLAST are the main predictors. To measure PSI-BLAST feature importance within VIPUR it first needs to be reverse engineered to make it work with Modeller or another tool which is able to implement mutations in PDB files. Once the reverse engineering is finished VIPURs feature importance is tested with shap\cite{} and or similar methods and software on the VTS.
%slundberg/shap: A unified approach to explain the output of any machine learning model, Slundberg
%Explaining Prediction Models and Individual Predictions with Feature Contributions, Štrumbelj
%Why Should I Trust You?": Explaining the Predictions of Any Classifier, Ribeiro
%Learning Important Features Through Propagating Activation Differences, Shrikumar
%Algorithmic Transparency via Quantitative Input Influence: Theory and Experiments with Learning Systems, Datta
%On Pixel-Wise Explanations for Non-Linear Classifier Decisions by Layer-Wise Relevance Propagation, Bach
%Interpreting random forests | Diving into data, Datadive

Currently SPVAA is not an automated method to predict pathogencity or deleteriouness of proteins, to transform it into an automated method the approach described in section \ref{subsec:GD_theoratical_large_scale_implementation} would be its guideline. 


%However it is at its current state it is a pipeline instead of a automated prediction tool that classifies protein variants. 

At the moment when its 


The cluster on which it is build does not optimal ran does not suffice to predict 
SPVAA is not feasible on the current cluster due to the lack of nodes and support for multi-node jobs which in both influence the amount of structures produced and therefore affect prediction accuracy. Two options are possible to improve the and has to use different software which is less resource intensive for predicting protein structures or it should only be used on larger clusters which have more nodes and support multi-node jobs. 



