\subsection{Monte Carlo method}
There are complex problems in a variety of research fields which could take up years or even centuries to compute with simple deterministic methods. For some problems there is an algorithm which makes it possible to cut down computation time significantly, but when no deterministic algorithm is available to speed up the process an empirical probabilistic method might be able to approximate the desired result. With the Monte Carlo method random samples are taken from the parameter space ,that describe a data set, and fed into a model which produces a potential outcome. By repeating the process more results are generated until at some point the data can display a pattern that describes the outcome. The result is a quantified probability which describes the chance that something might occur based on the quantity of occurrence generated by the model \cite{stephanie_monte_2015,wikipedia_monte_2019,wikipedia_monte-carlosimulatie_2018}.
%Monte Carlo Simulation / Method, Stephanie
%Monte Carlo method, Wikipedia
%Monte-Carlosimulatie, Wikipedia
\newline
The Monte Carlo methods can differ depending on the algorithm and application in which it is used, but in summary most implementations will follow a general pattern \cite{wikipedia_monte_2019}:
%Monte Carlo method, Wikipedia
\begin{enumerate}
	\setcounter{enumi}{-1}
	\item Construct a model which is able to describe an outcome of the problem.
	\item Define the space of which inputs can be used by the model to get an outcome (creating a parameter space). 
	\item Use the model to generate results based on random sampled input from the parameter space.
	\item Order and determine which results are part of a certain outcome and draw conclusions on the generated statistical evidence.
\end{enumerate}

\label{subsec:Monte_Carlo_Method}

\subsection{The use of the Monte Carlo method and its pitfalls}
The Monte Carlo method is widely used within various applications in different fields of science but it is limited in the type of problems it can solve and is suitable for; problems of which all the inputs are known but it is too inefficient to compute deterministically; situations that require uncertainty to be incorporated into the analysis and exploring parameters for a model that give a better impact than the current parameters. The mentioned type of problems it can solve all tend to rely on significant quantities of data which makes it a relative time consuming process for generating results. Meaning of the generated result is highly depended on the model and random sampling techniques which both contribute to an errors in the result \cite{stephanie_monte_2015,wikipedia_monte_2019,alon_honig_introduction_nodate}.
%Monte Carlo Simulation / Method, Stephanie
%Monte Carlo method, Wikipedia
%Introduction to Monte Carlo Methods, Alon Honig