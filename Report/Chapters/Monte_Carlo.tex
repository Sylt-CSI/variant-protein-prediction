\subsection{Monte Carlo Method}
There are complex problems in a variety of research fields which could take up years or even centuries to compute with simple deterministic methods, in some cases there is an algorithm which makes it possible to cut down computation time significantly,  problems that have no such algorithm the Monte Carlo method might be of use. With the Monte Carlo method random samples are taken from the parameter space ,that describe a data set, and fed into a model which produces a potential outcome. By repeating the process more results are generated until at some point the data can start to display a pattern of what the outcome can be that was out of scope by a deterministic method. It results in a quantified probability which describes the chance that something might occur based on the quantity of occurrence generated by a model that is able to produce a potential outcome \cite{}.
%Monte Carlo Simulation / Method, Stephanie

The Monte Carlo method has various algorithmic implementations that 
Monte Carlo methods can be implemented differently depending on the problem but in general follow a pattern:
\begin{enumerate}[1:]
	\item Construct a model which is able to describe an outcome of the problem.
	\item Define the space of which inputs can be used by the model to get an outcome (creating a parameter space). 
	\item Use the model to generate results based on random sampled input from the parameter space and store the result.
	\item Determine which results are part of .
	\item 
\end{enumerate}

\subsection{Monte Carlo Pitfalls}
\label{Subsec:Monte_Carlo_Pitfalls}


Monte Carlo method has two factors that are important to determine the probability:
\begin{enumerate}[1:]
	\item All the points must be distributed uniformly, otherwise the prediction will have limited meaning.
	\item Large quantities are recommend since it improves the resolution of the answer.
\end{enumerate}
