\subsection{Monte Carlo Method}
There are complex problems in a variety of research fields which could take up years or even centuries to compute with simple deterministic methods. For some problems there is an algorithm which makes it possible to cut down computation time significantly, but when no deterministic algorithm is available to speed up the process an empirical probabilistic method might be able to approximate the desired result. With the Monte Carlo method random samples are taken from the parameter space ,that describe a data set, and fed into a model which produces a potential outcome. By repeating the process more results are generated until at some point the data can display a pattern that describes the outcome. The result is a quantified probability which describes the chance that something might occur based on the quantity of occurrence generated by the model \cite{}.
%Monte Carlo Simulation / Method, Stephanie
%Monte Carlo method, Wikipedia
\newline
\newline
The Monte Carlo methods can differ depending on the algorithm and application in which it is used, but in summary most implementations will follow a general pattern \cite{}:
%Monte Carlo method, Wikipedia
\begin{enumerate}
	\setcounter{enumi}{-1}
	\item Construct a model which is able to describe an outcome of the problem.
	\item Define the space of which inputs can be used by the model to get an outcome (creating a parameter space). 
	\item Use the model to generate results based on random sampled input from the parameter space.
	\item Order and determine which results are part of a certain outcome and draw conclusions on the generated statistical evidence.
\end{enumerate}

\label{subsec:Monte_Carlo_Method}

\subsection{The use of the Monte Carlo method and its pitfalls}
The Monte Carlo method has a limited scope of problems it can solve, and is suitable for; problems of which all the inputs are known but it is too inefficient to compute deterministically; situations that require uncertainty to be incorporated into the analysis; exploring parameters for a model that give a better impact than the current parameters and transforming complex to simpler systems. 

it is unsuitable for; giving simple output in complex systems since there is such a wide variety of parameters that can make up the results; finding realistic predictions since the whole point of the method is finding potential outcomes not a definitive answer;

When the method is used for the wrong type of data

If the model is incorrect or the use case 

Not all types of problems are solvable with the Mont

Not every method is applicable to every problem which is also with the Monte Carlo method 


and with that the Monte Carlo method also has it strengths and weaknesses.  therefor it also has its strengths in pr
The Monte Carlo method is widely used within various applications in different fields of science and help But not all problems are suitable, 


They can be used for problems that are too complex for deterministic approaches but in which all parameters are known that can contribute to an outcome. In some situations uncertainty is a d

 but are good for only certain types of problems. They can be used a 
Specific types of problems that can be solved by the Monte Carlo method if no algorithm is

While Monte Carlo has many applications in different fields of science and is good for solving problems

 it is also has its limitations, it s

 in what it can do and is prone to errors

prone to large errors depending on the problem what it is used for. 



\label{subsec:Monte_Carlo_Pitfalls}


Monte Carlo method has two factors that are important to determine the probability:
\begin{enumerate}[1:]
	\item All the points must be distributed uniformly, otherwise the prediction will have limited meaning.
	\item Large quantities are recommend since it improves the resolution of the answer.
\end{enumerate}
