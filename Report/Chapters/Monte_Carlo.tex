\subsection{Monte Carlo Method}
There are complex problems in a variety of research fields which could take up years or even centuries to compute with simple deterministic methods. For some problems there is an algorithm which makes it possible to cut down computation time significantly, but when no deterministic algorithm is available to speed up the process an empirical probabilistic method might be able to approximate the desired result. With the Monte Carlo method random samples are taken from the parameter space ,that describe a data set, and fed into a model which produces a potential outcome. By repeating the process more results are generated until at some point the data can display a pattern that describes the outcome. The result is a quantified probability which describes the chance that something might occur based on the quantity of occurrence generated by the model \cite{}.
%Monte Carlo Simulation / Method, Stephanie
%Monte Carlo method, Wikipedia
\newline
\newline
The Monte Carlo methods can differ depending on the algorithm and application in which it is used, but in summary most implementations will follow a general pattern \cite{}:
%Monte Carlo method, Wikipedia
\begin{enumerate}
	\setcounter{enumi}{-1}
	\item Construct a model which is able to describe an outcome of the problem.
	\item Define the space of which inputs can be used by the model to get an outcome (creating a parameter space). 
	\item Use the model to generate results based on random sampled input from the parameter space.
	\item Order and determine which results are part of a certain outcome and draw conclusions on the generated statistical evidence.
\end{enumerate}

\label{subsec:Monte_Carlo_Method}

\subsection{The use of the Monte Carlo method and its pitfalls}
Specific types of problems that can be solved by the Monte Carlo algorithm is

While Monte Carlo has many applications in different fields of science and is good for solving problems

 it is also has its limitations, it s

 in what it can do and is prone to errors

prone to large errors depending on the problem what it is used for. 

good

examining complex aggregations from simple actions (analyzing dropping grain of sand) good for when we can find all actions influencing the outcome, but when unsure about the outcome
incorporate uncertainty into analysis
explore intuitive results (finding better drivers for)
simplify complex systems (likelihood of breakage, how much water can flow through the system)
what if situations

bad

simple action for complex aggregations (complex results and describe simple things) whe sure about the outcome but not knowing all the inputs
finding realistic results (accuracy)
robustness (changing values??)
speed of execution (It is a process of trial an error)
finding reasonable scenarios
measuring the likelihood of a scenario


\label{subsec:Monte_Carlo_Pitfalls}


Monte Carlo method has two factors that are important to determine the probability:
\begin{enumerate}[1:]
	\item All the points must be distributed uniformly, otherwise the prediction will have limited meaning.
	\item Large quantities are recommend since it improves the resolution of the answer.
\end{enumerate}
