\subsection{Two methods: scale and detail}
VIPUR is a machine learning approach for predicting pathogenicity of proteins. The 106 features that were used for machine learning originate mainly (94\%) from the Rosetta software suite (Section \ref{subsec:MM_Rosetta}) applications; DDG monomer (Section \ref{subsubsec:MM_DDG_Monomer}), Relax (Section \ref{subsubsec:MM_Relax}) and Rescore (Section \ref{subsubsec:MM_Rescore}), the remaining features were collected from PSI-BLAST (Section \ref{subsec:MM_PSI_BLAST}) and Probe (Section \ref{subsec:MM_Probe}). 
All proteins in the VTS of which structures were known or had fragments available were collected from Modbase \cite{} and
% modbase, a database of annotated comparative protein structure models and associated resources, pieper
SWISS-MODEL \cite{}.
% Automated comparative protein structure modeling with SWISS-MODEL and Swiss-PdbViewer: A historical perspective, Guex
% Modeling protein quaternary structure of homo- and hetero-oligomers beyond binary interactions by homology, Bertoni
% Toward the estimation of the absolute quality of individual protein structure models, Benkert
% The SWISS-MODEL Repository—new features and functionality, Bienert
% SWISS-MODEL: homology modelling of protein structures and complexes, Waterhouse
Proteins that did not have a structure within the databases were modeled with Modeller (Section \ref{subsec:MM_Modeller} based on protein fragments that had the highest amino acid sequence identity to the protein.
In some experimental determined structures duplicate chains, ligands, metals and non-standard amino acids were present, these inconsistencies are able to alter the features generated by software and could in some case hinder feature collection, therefor they were removed to make the data homogeneous. Structural mutations of proteins that are in the VTS were introduced by a script using PyMOL (Section \ref{subsec:MM_PyMOL}) by default or PyRosetta (Section \ref{subsec:MM_PyRosetta}) if PyMOL was not available.

%Something about that we tried to use the VTS and add our protein for information.


Another approach for determining pathogenicity of a mutation is by assessing energy differences between a wild type and mutant protein residues inside its complex. Analyzing mutations from this perspective gives the ability to view a complex in whole and determine how residues cause perturbations in a complex. Missense mutations in monomers of complexes were made with Modeller (Section \ref{subsec:MM_Modeller}) and the backbone was refined with Rosetta's backrub application (Section \ref{subsubsec:MM_Backrub}), to lower the energy levels within side chains Rosetta relax (Section \ref{subsubsec:MM_Relax}). This method shows similarities to that of VIPUR,  was tested with TNFRSF1A (Section \ref{subsec:CD_TNFRSF1A}) and its ligands TNF $\alpha$ and $\beta$. This method keeps: duplicate chains ligands and metals within the structure, water is excluded since it can cause issues with Rosetta tools (Section \ref{subsec:MM_Rosetta:}).

\subsection{Rosetta}
Rosetta is a software suite that has a variety of tools that are developed to aid in macro molecular and antibody ,analysis, design and modeling \cite{}.
% About | RosettaCommons, Rosetta Commons
Both approaches rely on the Relax (Section \ref{subsubsec:MM_Relax}) for minimizing side chains. VIPUR uses rescore (Section \ref{subsubsec:MM_Rescore}) to acquire information about protein structures.

Both methods rely on Relax  to minimize energies in the side chains of the remodeled structures. With  DDG monomer  both rely on energy minimization's  in the side chains of the protein structures and need to information on energy changes in 

The scores generated for the machine learning within the VIPUR approach rely on results generated by Rosetta software and to apply this approach the steps are reproduced.  
Several strategies were employed for realizing mutated structures, the first strategy was to identify the whole structure of proteins


 The initial structure of the protein was produced with the application abinitio relax. For the prediction the application requires an amino acid sequence to identify homologous sequences in a curated database. Homologous sequences within the database are found by the BLAST algorithm, when a

For the search of the sequences it uses the BLAST algorithm and to find homologous amino acid sequences which have protein structures.

requires an amino acid sequence and  it takes an amino acid sequence as input and searches in a curated protein database BLAST for finding homologous sequences. 


to align sequences with to acquire homologous sequences.  The homologous With these sequences it finds structures related to the protein
For the prediction of the initial structure of TNFR the application abinitio relax was used. 

With this tool a sequence is inserted as input that is aligned to 


Missense mutated proteins have an altered amino acid that can cause differences in interactions with other amino acids, which can influence the backbone or side chain positions of a protein and therefore affect the structure. Software that makes missense mutations in protein structures (Modeller, PyMOL, PyRosetta) tend to replace residues without optimizing, causing odd energy levels or steric hindrance to arise.
\label{subsec:MM_Rosetta}
\newline
\textit{Rosetta software suite Version 3.10}

\subsubsection{Relax}
Relax was the only application used by both methods which tried to minimize energies in local conformational search space \cite{} within the mutated structures. From each minimization attempt the structure was saved and scores for certain properties were calculated and written into a single file. From this score file VIPUR collected all samples and made 83 features out of it, the detailed approach used the scores from a single structure for its assessment.
%Relax application, Rosetta Commons 
\label{subsubsec:MM_Relax}

\subsubsection{DDG Monomer}
DDG monomer is meant to predict energetic stability of a point mutation in  monomeric protein.The application was used by VIPUR to collect features related to energies and hydrogen, disfulfide, bonds and constraints differences between the wild type and a mutated protein. To execute the tool a script had to be ran that renumbers the wild type pdb file and it requires a "mutation file" that describes the change of a residue based on name and position changes to a different residue \cite{}.
%ddg_monomer application, Andrew Leaver-Fay
%Backrub application, Smith
\label{subsubsec:MM_DDG_Monomer}

\subsubsection{Rescore}
With this tool Rosetta scores can be calculated based on silent or PDB files proteins structures \cite{} , the output is identical to that is written within the score files produced by Relax (Section \ref{subsubsec:MM_Relax}).
%Score Commands, Jared
\label{subsubsec:MM_Rescore}

\subsubsection{Backrub}
The backrub application is based on the Monte Carlo method (Section \ref{section:Chap_Monte_Carlo}), and alters a protein by moving its backbone residues with a strategy called fix end move (FEM). With this strategy, groups of residues are selected at random from the structure, it can contain up to: four dihedral, two bond angles and two end points. Both ends of a group are fixated at their position in which a new angle $\alpha$ arises, within this angle residues are pivoted in their natural occurring maximum range of $\pm \ang{10}$ \cite{}.
%Efficient Monte Carlo trial moves for polypeptide simulations, Betancourt
With this method the backbones of newly introduced mutations were altered, for each attempt a new file was generated and a score was written to a score file, from which the lowest Rosetta scoring was selected to be further relaxed (Section \ref{subsubsec:MM_Relax}).

It was used on the mutated protein to relax the modified backbone structure.
\label{subsubsec:MM_Backrub}

\subsection{PyRosetta}
Is an application programming (API) which has Python bindings (Section \ref{subsec:MM_Python}) for the Rosetta software suite (Section \ref{subsec:MM_Rosetta}), it founds its use in VIPUR when no PyMOL (Section \ref{subsec:MM_PyMOL}) was available to mutate residues within a structure \cite{}.
%PyRosetta, Jeffry
\label{subsec:MM_PyRosetta}
\newline
\textit{Version 4}

\subsection{PSI-BLAST}
Position specific iterative basic local alignment search tool (PSI-BLAST) focuses on distant relatives of proteins by making a profile of the sequence and querying it at a protein sequence database. With the generated results a new profile is constructed and queried again, these steps are repeated several times to determine which residues are found in relatives of the protein. The result is a position specific scoring matrix (PSSM) which describes the frequency of which residues are substituted by a specific other residue, positive is more, negative is less common \cite{}.
%PSIBLAST, NCBI
%PSSM Viewer, NCBI
%BLAST, Wikipedia 
From the PSSMs sequences features were acquired for the VIPUR machine learning method.
\label{subsec:MM_PSI_BLAST}
\newline
\textit{Position-Specific Iterated BLAST 2.7.1+}

\subsection{Probe}
Probe is able to evaluate atom packing for a single protein or interacting proteins by creating a probe, which is described as a sphere like object, that marks an area with dots when at least two non-covalent atoms are in contact with the probe at the same position \cite{}. VIPUR used this tool to calculate solvent accessible surface area (SASA or ACCP).
%Visualizing and quantifying molecular goodness-of-fit: small-probe contact dots with explicit hydrogen atoms11Edited by J. Thornton,Word
%Probe Software : Kinemage Website, Richardson LAB
\label{subsec:MM_Probe}
\newline
\textit{version 2.16.130520}

\subsection{Robetta prediction server}
The web tool Robetta integrates several tools to form protein structures based on sequence alignments of previously discovered structures also known as homology modeling (Section \ref{}). It requires an amino acid sequence, optionally constrains and fragments can be added to disallow movement of certain structures or add known fragments to avoid calculating pieces that are already known. With this information Robetta search with the help of sequence aligners for known fragments and tries to incorporate them into a single protein structure \cite{}.
%High-Resolution Comparative Modeling with RosettaCM, Song
%Protein homology detection by HMM-HMM comparison, Soding
%Template-based protein structure modeling using the RaptorX web server,Källberg
%Improving protein fold recognition and template-based modeling by employing probabilistic-based matching between predicted one-dimensional structural properties of query and corresponding native properties of templates, Yang
%Protein structure determination using metagenome sequence data,Ovchinnikov
The used structures of TNFRSF1A (Section \ref{section:Chap_Cell_Death}) were in complete and could therefore lack information regarding the structure when a mutation is introduced. To form a whole protein the, fragments of several known structures are joined by the Abinitio protocol within act on th with this web tool it was possible to predict the missing pieces of the protein
\label{subsubsec:MM_Robetta}

\subsection{I-TASSER prediction server}
The I-TASSER web server is a tool that is able to predict protein structures with a FASTA sequence. The first step it takes is finding structural templates which resemble the sequence by local meta-threading server (LOMETS). LOMETS starts with multiple sequence alignment of which several sequences will undergo protein threading by different programs to form structural templates. The templates are assessed based on the highest alignment Z-score, the program specific confidence score and sequence identity \cite{}.
%LOMETS, Zhang LAB
%LOMETS: A local meta-threading-server for protein structure prediction, Wu
The known fragments of TNFRSF1A (Section \ref{section:Chap_Cell_Death}) were given as a template to I-TASSER and modeled into a whole protein to make it possible to introduce mutations and predict pathogenicity of a variants.
\label{subsec:MM_I_TASSER}
\newline
\textit{Server version}

\subsection{SLURM}
For computational jobs where a laptop or desktop does not suffice because due to the lack computational resources a computer cluster could come to aid. These clusters consist out of several computers that execute resource intensive tasks, to manage these systems for many clients and to use these clusters optimal a workload manager mlike simple Linux utility resource management (SLURM), is installed. Jobs are submitted that request resources for execution and are scheduled on the systems queue which is ordered based on priorities, resource requirement and time.
\label{subsec:MM_SLURM}

\subsection{Modeller}
Modeller is software that is developed for homology modeling but it was used for its utilities which allowed to; complete protein data bank (PDB) structures with missing atoms; predict disulfide bonds that were missing and mutate protein residues \cite{}. 
%About MODELLER, Modeller
%Comparative Protein Structure Modeling Using Modeller, Eswar
\label{subsec:MM_Modeller}
\newline
\textit{Version 9.21}

\subsection{GAVIN Machine Learning Data Table}
Is a collection of nucleotide mutations from rare diseases used by the GAVIN \cite{} machine learning approach. From this set the genes of TNFRSF1A (Section \ref{section:Chap_Cell_Death}) with a missense mutation were filtered (Section \ref{subsec:MM_R}) and written into a format which the variant effect predictor could (VEP) \cite{} could read and translate from nucleotide to protein mutations. The classification of these variants was according to Clinvar significance values \cite{}. 
%GAVIN: Gene-Aware Variant INterpretation for medical sequencing, Van der Velde
%Variant Effect Predictor - Homo sapiens - GRCh37 Archive browser 96, ensemble
%Representation of clinical significance in ClinVar and other variation resources at NCBI, NCBI
\label{subsec:MM_GAVIN_data_table}

\subsection{GenomAD}
The GenomAD database consists of unified data from large scale genome sequencing data projects and is based on genome reference consortium human genome build 37 human genome 19 (GRCh37/hg19). From this database missense mutations were collected for TNFRSF1A (Section \ref{section:Chap_Cell_Death}), no classification was known for these mutations \cite{}.
%gnomAD,gnomAD
\label{subsec:MM_GenomAD}

\subsection{Infevers}
Is a website about hereditary auto immune diseases with for each disease a downloadable table about the known mutations and their classification. The table for TRAPS disease (Section \ref{section:Chap_Cell_Death}) was used to collect missense mutations of TNFRSF1A gene \cite{}.
%INFEVERS: the Registry for FMF and hereditary inflammatory disorders mutations, Sarrauste de Menthière
\label{subsec:MM_Infevers}

\subsection{Research Collaboratory for Structural Bioinformatics}
Research Collaboratory for Structural Bioinformatics (RCSB) is a database where whole or fragmented experimentally determined proteins structures that are published can be found and downloaded. The Fragments for modeling (Sections \ref{subsec:MM_Rosetta}, \ref{subsec:MM_I_TASSER}) whole TNFRSF1A (Section \ref{section:Chap_Cell_Death}) (1EXT \cite{}) and determining the differences in energy levels (Section \ref{subsubsec:MM_Relax}) with TNF $\beta$ (1TNR \cite{}) with the interaction site were acquired from this database \cite{}.
%Crystal structure of the soluble human 55 kd TNF receptor-human TNFβ complex: Implications for TNF receptor activation, Banner
%Structures of the extracellular domain of the type I tumor necrosis factor receptor,Naismith
%RCSB Protein Data Bank: Sustaining a living digital data resource that enables breakthroughs in scientific research and biomedical education, Burley
\label{subsec:MM_RCSB}

\subsection{Uniprot}
Knowledge from various omic domains about proteins has been linked together into single database called Uniprot which makes all information accessible at once, for TNFRSF1A (Section \ref{section:Chap_Cell_Death}) the FASTA sequences were collected from Uniprot and for structures it redirected to (Section \ref{subsec:MM_RCSB}) \cite{}.
%UniProt: a hub for protein information,The UniProt Consortium
\label{subsec:MM_Uniprot}

\subsection{R scripting language}
With R the tables from GenomeAD, GAVIN and Infevers (Sections \ref{subsec:MM_GenomAD} \ref{subsec:MM_GAVIN_data_table} \ref{subsec:MM_Infevers}) of TNFRSF1A missense mutations (Section \ref{section:Chap_Cell_Death}) were merged together in a new comma seperated values file with their known classifications. Ordering and filtering the double mutations and removing double classifcations where done with Python (Section \ref{subsec:MM_Python}).
\label{subsec:MM_R}
\newline
\textit{R scripting front-end version 3.5.2 (2018-12-20)}

\subsection{Bash}
Unix like operating systems (OS) have a shell which allows users to interact with programs on a computer or with the computer itself based on commands submitted.  The default shell for MacOS and also for several Linux distributions is the Bourne again shell (Bash) which was used to launch Python scripts (Section \ref{subsec:MM_Python}) and submit jobs to the SLURM workload manager (Section \ref{subsec:MM_SLURM}). 
\label{subsec:MM_Bash}
\newline
\textit{Laptop Version GNU bash, version 3.2.57(1)-release (x86\_64-apple-darwin18)}
\newline
\textit{Server Version GNU bash, version 4.1.2(2)-release (x86\_64-redhat-linux-gnu)}

\subsection{Python}
Both VIPUR and the pipeline that minimizes backbone (Section \ref{subsubsec:MM_Backrub}) and side chain energies (Section \ref{subsubsec:MM_Relax}) were written in Python due to its capabilities, ease of use and because modeller (Section \ref{subsec:MM_Modeller}) for MacOS relies on the system version of Python and does currently not support newer versions besides the one found within the OS of Mac. The mutations that were put together from the different tables (Sections \ref{subsec:MM_GenomAD}, \ref{subsec:MM_Infevers}, \ref{subsec:MM_GAVIN_data_table}) with R {Section \ref{subsec:MM_R}) were filtered by a Python script. To apply each mutation correctly on the proteins in the detailed method a script was written in which files were generated that described in a compact format on which chains and position a mutation resided.
\label{subsec:MM_Python}
\newline
\textit{Laptop version 2.7.15}
\newline
\textit{Server version 2.7.11}

\subsection{MPI}
Some tools from the Rosetta software suite (Sections \ref{subsec:MM_Rosetta}) have the ability to use multiple central processing unit (CPU) cores from a single computer or from multiple computers. With a message parsing interface (MPI) it is possible for software to communicate between CPU cores on the same and on different computers to exchange information about processes and therefor solving solutions faster.
\label{MM_MPI}
\newline
\textit{OpenMPI/1.8.8-GNU-4.9.3-2.25}
\newline

\subsection{PyMOL}
Visualization of 3D structures, making images of proteins, putting the known orientations of monomeres in position and replacing TNF $\beta$ with TNF $\alpha$ were done in PyMOL \cite{}, also PyMOL had some Python bindings to mutate proteins, which were used by VIPUR.
% PyMOL | pymol.org, Schrödinger
\label{subsec:MM_PyMOL}
\newline
\textit{Version 2.2.3}